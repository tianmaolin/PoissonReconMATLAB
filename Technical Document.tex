\documentclass[11pt,twoside]{article}
\usepackage{amsmath,latexsym,amssymb,amsfonts,amsbsy}
\usepackage{graphicx}

\textwidth 16cm \textheight 22cm \oddsidemargin 0in
\evensidemargin 0in
%
\newfont{\bb}{msbm10}
\def\Bbb#1{\mbox{\bb #1}}

\def\diag{{\rm diag}}
\def\Diag{{\rm Diag}}
\def\tridiag{{\rm tridiag}}
\def\det{{\rm det}}
\def\sign{{\rm sign}}
\def\opt{{\rm opt}}
\def\rnm{\mathbb{R}^{n\times m}}
\def\rmn{\mathbb{R}^{m\times n}}
\def\cmn{\mathbb{C}^{m\times n}}
\def\cnm{\mathbb{C}^{n\times m}}
%
\def\cnn{\mathbb{C}^{n\times n}}
\def\rnn{\mathbb{R}^{n\times n}}

\def\exp{{\rm exp}}
\def\v{}
\def\vect{}
\newtheorem{algorithm}{Algorithm}[section]
\newtheorem{method}{Method}[section]
\newtheorem{example}{Example}[section]
\newtheorem{proposition}{Proposition}[section]
\newtheorem{theorem}{Theorem}[section]
\newtheorem{definition}{Definition}[section]
\newtheorem{lemma}{Lemma}[section]
\newtheorem{corollary}{Corollary}[section]
\newcommand{\vecspace}{\bf C}
\newcommand{\IC}{\makebox{{\Bbb C}}}
\newcommand{\IR}{\makebox{{\Bbb R}}}
\newcommand{\leals}{\makebox{{\Bbb L}}}
\newcommand{\neals}{\makebox{{\Bbb N}}}
\newcommand{\Z}{\makebox{{\Bbb Z}}}
\newcommand{\Frac}{\displaystyle\frac}
\newcommand{\Sum}{\displaystyle\sum}
\newcommand{\Lim}{\displaystyle\lim}
\newcommand{\Max}{\displaystyle\max}
\newcommand{\Prod}{\displaystyle\prod}
\newcommand{\Min}{\displaystyle\min}
%\newcommand{\Bigl}{\displaystyle\{ }
%\newcommand{\Bigr}{\displaystyle\} }
\newcommand{\compar}[1]{\langle {#1} \rangle}
\newcommand{\ccompar}[1]{\langle\langle {#1}\rangle\rangle}
\def \endproof{\vrule height8pt width 5pt depth 0pt}
\baselineskip=14pt
\parindent=0pt
\parskip=3pt
\overfullrule=0pt
\renewcommand\theequation{\arabic{equation}}
\renewcommand\thefigure{\arabic{figure}}
\begin{document}
\cleardoublepage \pagestyle{myheadings}

\bibliographystyle{plain}

\title{\bf Technical Document: Poisson Surface Reconstruction in MATLAB}


\author{Maolin Tian\\
{\it School of Mathematical Science}\\
{\it Shanghai Jiao Tong University}\\
{\it Email: tml10016@163.com }\\[2mm]
}

\maketitle \markboth{\small  Maolin Tian}
{\small  Technical Document of Poisson Surface Reconstruction in MATLAB }

\begin{abstract}
The details of Poisson surface reconstruction for computing in MATLAB.
\end{abstract}

\bigskip

\section{Poisson Reconstruction}\label{PR-sec}
The Poisson equation \cite{PR} is 
\begin{equation}\label{dbXV}
\Delta \chi  = \nabla \cdot \vec{V}.
\end{equation}
Let the smoothing filter \(F\) be B-spline. 
That is,
\begin{equation}\label{FFs}
F_p (q) = B(\frac{q-p}{w}) \frac{1}{w^3},
\end{equation}
\begin{equation}\label{BBBB}
B(x,y,z) = (B(x)B(y)B(z))^{*n},
\end{equation}
and
\begin{equation}\label{Bt10}
B(t) = \left\{
\begin{aligned}
& 1 , |t| < 0.5 \\
& 0 , \text{otherwise}.
\end{aligned}
\right.
\end{equation}
The sampling density is estimated by
\begin{equation}\label{sampling_density}
W (q) = \sum_{s \in S} F_{s.p} (q).
\end{equation}
The vector field is 
\begin{equation}\label{dXFN_discrete}
\vec{V}(q) = \sum_{s \in S}  F_{s.p} (q) \  s.\vec{N} \  \frac{1}{W(s.p)}.
\end{equation}

Assume the solution 
\begin{equation}\label{XxB}
\chi(p)=\sum_{i=1}^N x_i F_i (p).
\end{equation}
By Galerkin method, the linear system from (\ref{dbXV}) is
\begin{equation}\label{dxBdVB}
-\langle \nabla \chi(p), \nabla F_i \rangle
=\langle \nabla \cdot \vec{V}, F_i \rangle,
\end{equation}
denoted by \(-Ax = b\), where
\begin{equation}\label{ABB}
A_{ij} = \langle \nabla F_i, \nabla F_j \rangle
\end{equation}
and
\begin{equation}\label{bVB}
b_i = \langle \vec V, \nabla F_i \rangle.
\end{equation}

\section{Computation in MATLAB}
The equations (\ref{sampling_density}), (\ref{ABB}) and (\ref{bVB}) in Section \ref{PR-sec} are multivariate.
Computationally, we need represent them in univariate form.
Here we consider the case of 2D, where input is 2-D points and output is the curve.
Let \(b(x)=B(x)^{*n}\) be the univariate B-spline with degree n.
Then \(B(x,y)\) is separable since from (\ref{BBBB})
\begin{equation}\label{Bxy_Bxn_Byn}
\begin{aligned}
B(x,y)
&= (B(x)B(y))^{*n}\\
&= (B(x))^{*n}\ (B(y))^{*n}\\
&= b(x)\ b(y).
\end{aligned}
\end{equation}
By (\ref{ABB}),
\begin{equation}\label{Aij_dFi_dFj}
\begin{aligned}
A_{ij}
&= \langle \nabla F_i, \nabla F_j \rangle\\
&= \int \nabla F_i \cdot \nabla F_j\, dp\\
&= \int \frac{\partial}{\partial x} F_i \frac{\partial}{\partial x}  F_j \, dp
+ \int \frac{\partial}{\partial y} F_i \frac{\partial}{\partial y}  F_j \, dp.
\end{aligned}
\end{equation}
Let \(i.c\) be the center of grid \(i\),
\(i.x\) is the first component and \(i.y\) is the second. 
By (\ref{FFs}),
\begin{equation}\label{Aij_bb_bb}
\begin{aligned}
& \int \frac{\partial}{\partial x} F_i \frac{\partial}{\partial x}  F_j \, dp \\
&= \int \frac{\partial}{\partial x} (B(\frac{p-i.c}{i.w})\frac{1}{i.w^2})\
\frac{\partial}{\partial x}  (B(\frac{p-j.c}{j.w})\frac{1}{j.w^2})\,  dp\\
&= \int \frac{\partial}{\partial x} B(\frac{p-i.c}{i.w})\frac{1}{i.w^3}\
\frac{\partial}{\partial x}  B(\frac{p-j.c}{j.w})\frac{1}{j.w^3}\,  dp\\
&= \int \int \frac{\partial}{\partial x} b(\frac{x-i.x}{i.w})b(\frac{y-i.y}{i.w})\frac{1}{i.w^3}\
\frac{\partial}{\partial x}  b(\frac{x-j.x}{j.w})b(\frac{y-j.y}{j.w})\frac{1}{j.w^3}\,  dx\,  dy\\
&= - \frac{1}{i.w^3\, j.w^3} \int b'(\frac{i.x-j.x-x}{i.w}) b'(\frac{x}{j.w})\,  dx
\int b(\frac{i.y-j.y-y}{i.w}) b(\frac{y}{j.w})\,  dy% \\
% &= \frac{1}{i.w^2\, j.w^2}\ (b' * b')(\frac{i.x}{i.w} - \frac{j.x}{j.w})\ (b * b)(\frac{i.y}{i.w} - \frac{j.y}{j.w}).
\end{aligned}
\end{equation}
The width is in \(\{2^{-d}|\,d\le D\}\), where \(D\) is max depth.
\(D\) is small, therefore the number of \(s.w,\,i.w,\,j.w\) is finite.
Let 
\begin{equation}b_w(t)=b(\frac{t}{w}),\end{equation}
\begin{equation}b'_w(t)=(b_w(t))'.\end{equation}
(\ref{Aij_bb_bb}) is 
\begin{equation}\label{xFixFjdp}
\begin{aligned}
& \int \frac{\partial}{\partial x} F_i \frac{\partial}{\partial x}  F_j \, dp \\
&= - \frac{1}{i.w^2\, j.w^2} (b_{i.w}'*b_{j.w}')(i.x-j.x)
\ (b_{i.w}*b_{j.w})(i.y-j.y)
\end{aligned}
\end{equation}

Similarly, (\ref{sampling_density}) can be computed by
\begin{equation}\label{WpsSFspp}
\begin{aligned}
W (p)
&= \sum_{s \in S} F_{s.p} (p)\\
&=\sum \frac{1}{s.w^2} b (\frac{x-s.x}{s.w}) \  b(\frac{y-s.y}{s.w}).
\end{aligned}
\end{equation}
And (\ref{bVB}) can be computed by (\ref{dXFN_discrete}) and
\begin{equation}\label{biVFidp}
\begin{aligned}
b_i &=\int \vec V \cdot \nabla{F}_i\,  dp\\
&=\int \sum_{s \in S}  F_{s.p} (q) \  s.\vec{N} \  \frac{1}{W(s.p)} \cdot \nabla{F}_i\,  dp\\
&=\sum_{s \in S} \frac{1}{W(s.p)} \int F_{s.p} \  s.\vec{N} \cdot \nabla{F}_i\, dp\\
&=\sum \frac{1}{W(s.p)} \  s.\vec{N} .x \int F_{s.p}\ \frac{\partial}{\partial x}\ F_i\, dp\\
&\  +\sum \frac{1}{W(s.p)} \  s.\vec{N} .y\int F_{s.p}\ \frac{\partial}{\partial y}\ F_i\, dp.
\end{aligned}
\end{equation}
Here \( s.\vec{N}.x\) is the first component of \(s.\vec{N}\).
As in (\ref{Aij_bb_bb}),
\begin{equation}\label{FspxFip}
\int F_{s.p}\ \frac{\partial}{\partial x} F_i\, dp
= \frac{1}{s.w^2 i.w^2} (b_{s.w}*b_{i.w}')(s.x-i.x)
\  (b_{s.w}*b_{i.w})(s.y-i.y)
\end{equation}

Hence, we can represent \( A_{ij},W(p),b_i\) by \(b(x)\),
and we need compute \(b_w*b_w,\, b_w*b_w',\, b_w'*b_w',\, b_w\).
It is easy to compute them, 
since the 4\(D\) univariate piecewise functions are simple.
We first compute a value table of them to accelerate.
Then get an approximate value from the table in computation.

\begin{thebibliography}{99}


 \bibitem{PR}
 Kazhdan, M., Bolitho, M., and Hoppe, H.
 {\em Poisson surface reconstruction}.
 In Proceedings of the fourth Eurographics symposium on Geometry processing (Vol. 7).



\end{thebibliography}

\end{document}
